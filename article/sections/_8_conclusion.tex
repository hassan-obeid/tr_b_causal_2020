\section{Conclusion}
\label{sec:conclusion}

In this chapter, we presented a short simulation aimed at highlighting the importance of using causal graphs and causal inference methods in transportation demand modeling endeavors.
This chapter was motivated by the existing disconnect between the fields of travel demand modeling
and causal inference.
Given the nature of transportation demand modeling efforts, one would expect that
causal inference methods have already been used in the field. To our surprise, there
has been very little mention of causal inference in the field of travel demand modeling.
\citet{brathwaite_2018_causal} have documented this disconnect and 
presented an initial framework for addressing it.

Following up on their paper, this paper presents a numerical exercising aiming to show
the importance of the data generating process in the estimation of effects of interest resulting from 
external "interventions". We showed, using the same utility specification model, that two data 
generating processes lead to bias in estimates of the effect of a policy intervention.

Data generating processes can be illustrated effectively using DAGs as shown by Judea Pearl.
We have shown that DAGs help the researcher clearly represent their assumptions about the
causal relationships within the data generating process. We presented a detailed process allowing researchers
to construct causal graphs based on their expert opinion of the problem at hand, more specifically
the variables of interest and their governing relationships. 

However, researchers should not assume that the causal graphs they constructed are "correct"
by default. These graphs carry many of the researchers beliefs on how the world operates.
These relationships are represented through the connections between the coded nodes in the causal graph.

Before researchers start with their modeling efforts on tangible problems, 
they should ensure the assumptions encoded in their proposed causal graphs are valid.
We present a method for testing the implications of the researcher's causal graph, both observed and latent.

Beyond testing the implications and assumptions encoded in a researcher's causal graph,
we describe why causal discovery is important in discerning relationships between covariates
within the data at hand and shortly present several causal discovery algorithms used by
other researchers in different fields. 
We then move to highlighting the use of one of these causal discovery algorithms on 
a simple example. We showed that causal discovery methods help us test independence within our data,
get a clearer picture about the uncertainty within the variables shown in the causal graph,
and expand the level and amount of tests performed on our data when compared to the tests
implicated by the structure of the researcher's proposed causal graph.

We then summarize one of the existing methods aimed at dealing with more realistic cases
than the one illustrated in the selection on observables simulation. We present examples of
latent confounding within the transportaion demand modeling field and highlight some examples
of current methods aiming at addressing problems resulting from confouding in certain situations.
We present a recently developed algorithm aiming to address the problems not addressed by previously
developed methods dealing with confouding. We use this model on the same example illustrated in
the selection on observables simulations and highlight the instances where it leads to a better
recovery of a confounder in one's causal graph. 

We then show that while the deconfounder algorithm might be a promising step in the right direction, 
it still has some notable deficiencies. We show that the deconfounder algorithms shows to be sensitive
to small errors in one's data.