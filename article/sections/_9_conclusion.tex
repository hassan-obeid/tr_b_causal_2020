\section{Conclusion}
\label{sec:conclusion}

Travel demand problems aim at forecasting the impact of proposed project and policies. These problems are causal in nature. 
To our surprise, there has been very little mention of causal inference in the field of travel demand modeling.
\citet{brathwaite_2018_causal} have documented this disconnect and presented an initial framework for addressing it.

In this chapter, we build up on \citet{brathwaite_2018_causal} and highlight the importance of using causal graphs and 
causal inference methods in transportation demand modeling efforts.

We presented a numerical exercise aiming to show the importance of  the data generating process in the estimation 
of effects of interest resulting from external interventions. 
We showed, using the same outcome model, that different assumptions about the data generating processes could 
lead to bias in estimates of the effect of interest.

Data generating processes can be illustrated effectively using DAGs as shown by \citet{pearl_1995_causal}.
We have shown that DAGs help researchers and practitioners represent their assumptions about the data generating process. 

We presented a process allowing researchers and practitioners to construct causal graphs based on their expert opinion of the problem at hand. 
We also presented methods for testing the implications encoded in the the practitioners proposed causal graph. 
These implications might be either observed, latent, or both.

Beyond testing the implications and assumptions encoded in a one's causal graph, we describe why causal discovery is important in discerning relationships between covariates one's data.
We shortly presented several causal discovery algorithms used by
other researchers in different fields. 
We highlighted the use of one of these causal discovery algorithms on a simple example. 
We showed that causal discovery methods help us test independence within our data, 
develop a clearer picture about the uncertainty within the variables shown in the causal graph, 
and expand the level and amount of tests performed on our data when compared to the tests implicated 
by the structure of the researcher's proposed causal graph.

We presented examples of latent confounding within the transportation demand modeling field and highlight 
some examples of current methods aiming at addressing problems resulting from confounding in certain situations.

We present a recently developed algorithm by \citet{wang_2019_blessings} aiming to address the problems not 
addressed by previously developed methods dealing with confounding. We use this model on the same example illustrated in
the selection on observables simulations and highlight the instances where it leads to a better recovery of 
a confounder in one's causal graph. 

We then show that while the deconfounder algorithm might be a promising step in the right direction, 
it still has some notable deficiencies. More specifically, we show that the deconfounder algorithm shows 
to be sensitive to small errors in one's data.
