\section{Conclusion}
\label{sec:conclusion}

Travel demand problems aim at forecasting the impact of proposed project and policies. These problems are causal in nature. 
To date, choice modelers have not routinized the use of causal inference techniques in the field of travel demand modeling.
\citet{brathwaite_2018_causal} documented this disconnect and presented an initial framework for addressing it.

In this chapter, we built upon their framework.
We highlighted the importance of using causaal graphs and causal inference methods in transportation demand modeling efforts and in choice modeling more generally.

We presented a numerical exercise aiming to show the importance of the data generating process in the estimation 
of treatment effects from external interventions. 
We showed, using the same outcome model, that different assumptions about the data generating processes could 
lead to bias in estimates of the effect of interest.

Data generating processes can be illustrated effectively using DAGs as shown by \citet{pearl_1995_causal}.
We have shown that DAGs help researchers and practitioners represent their assumptions about the data generating process. 

We presented a process for constructing causal graphs based on their expert opinion of the problem at hand. 
We also presented methods for testing the implications encoded in a proposed causal graph. 
These implications might be either observed, latent, or both.

Beyond testing the implications and assumptions encoded in a causal graph, we describe why causal discovery is important in discerning relationships between one's covariates.
We briefly presented several causal discovery algorithms used by
researchers in different fields. 
Following the review, we demonstrated the use of one of these causal discovery algorithms, the PC algorithm, on a simple example. 
We showed that causal discovery methods help us test independence within our data, 
develop a clearer picture about the uncertainty within the variables shown in the causal graph, 
and expand the level and amount of tests performed on our data when compared to the tests implicated 
by the structure of the researcher's proposed causal graph.

We presented examples of latent confounding within the transportation demand modeling field and highlight 
some examples of current methods aiming at addressing problems resulting from confounding in certain situations.

We present a recently developed algorithm by \citet{wang_2019_blessings} aiming to address the problems not 
addressed by previously developed methods dealing with confounding. We use this model on the same example illustrated in
the selection on observables simulations and highlight the instances where it leads to a better recovery of 
a confounder in one's causal graph. 

We then show that while the deconfounder algorithm might be a promising step in the right direction, 
it still has some notable deficiencies. More specifically, we show that the deconfounder algorithm shows 
to be sensitive to small errors in one's data.

We hope that this chapter has provided enough preliminary evidence to demand modelers on the importance of using DAGs in their analyses.
Using DAGs will allow modelers to clearly outline their assumptions about their own problems and will allow them to track their analysis workflow more easily.
By now, choice modelers should have preliminary knowledge (and additional references) on how to:
\begin{itemize}
    \item Construct DAGs that fit their specific contexts
    \item Test the robustness and credibility of produced DAGs against available data
    \item Make use of causal discovery techniques to further assess the robustness of the assumptions built into our causal graphs and make new causal graphs that are based on the data at hand
\end{itemize}

We expect that this chapter has motivated demand modelers to step outside of their traditional analytical frameworks and make use of the techniques presented across the causal inference literature.
Moreso, we hope that this introduction will push demand modelers to contribute to and advance the causal inference literature.