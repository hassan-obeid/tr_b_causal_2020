\section{Overview of Causal Graphs}

\subsection{Prior Uses of Causal Graphs in Choice Modelling}

With the introduction to causal graphical models given above, we (the authors) do not want readers to think that we believe the entire notion of a causal graph is foreign to choice modelling.
We do not think this is true.
We know that choice modellers have made use of causal graphs in various forms for years.
Here are two examples to illustrate our point.
First, consider the case of Random Utility Maximization (RUM) models.

For decades now, choice modellers have used stylized directed graphical models to denote or convey the meaning of their economic framework's notion of causality in the context of their research.
As an example, take the following figure from Walker and Ben-Akiva \cite authors.
In Figure \ref{figure_title} Walker and Ben-Akiva depict the assumption that covariates X cause random utilities U which collectively cause the choice via utility maximization.
The reason I call such diagrams stylized is because of their lack of detail of the relationships within the columns of the design matrix X.
When speaking of the covariates collectively, one is unable to distinguish whether one covariate causes another.
Such intra-covariate relationships are important for estimating the causal effects of a given policy, as demonstrated in Section \ref{Sec_label}.

Another area of relation with prior choice modelling literature is in Integrated Choice and Latent Variable (ICLV) models.
These models make inference on one or more latent variables, alongside the other parameters in one's model.
An example of such ICLV models is in Figure \ref{}, based on Figure \_bla\_ of Walker and Ben-Akiva \cite.
As with the causal graphs of RUM models, ICLV causal graphs depict a particular set of structural modeling assumptions about the data generating process.

In both RUM and the ICLV diagrams, the main distinction between the causal graphs of Pearl \citet{article_name} and the causal diagrams in econometrics are two two-fold.
First, RUM and the ICLV causal graphs ignore relations between the explanatory variables.
Typically, choice modelling causal graphs depict all explanatory variables together as if they are in independently generated groups that then affect one's utilities and choices.
In the language used by a large group of causal inference scholars, econometric causal diagrams ignore the treatment assignment mechanism.

Secondly, causal graphs in choice modelling papers are purely didactic.
They convey how choice modellers perceive the world and the choice generation process.
However, they are seldom treated as a fully fledged model that has empirical implications that merit verification.
In this sense, choice modellers ignore the efforts from causal inference researchers in computer science.
There, researchers spend much effort testing their causal graphs to see if the data supports their graphs implications.

In conclusion, choice modellers have long made use of causal graphs in select contexts to convey causal assumptions about choice processes.
Thus far, however, we have underutilized these tools.
We rarely use causal graphs to encode assumptions about how our explanatory variables came to be in choice situations, and we do not routinely test causal graphs against empirical choice data.
These two issues represent opportunities for the field of choice modelling to gain from the insights and work of those who study causal inference.
The rest of the chapter will focus on how we can construct causal diagrams that pay attention to both explanatory and outcome variables, how we can test a given graph against one's data, and how one can deal with "real-world" graphs which frequently contain some sort of unobserved confounding.
