\section{Overview of Causal Graphs}

Causal diagrams and causal graphical models have been introduced by Pearl 
(2000) as a powerful tool for causal inference, especially in observational 
studies. Perhaps one of the most important and useful features of causal 
graphs when dealing with causal inference problems is the clear illustration 
of the causal relationships between the variables. It is important to note 
that the graphical approach to causality focuses primarily on issues of 
identification of causal effects, that is, given a directed acyclic graph (
DAG) that encodes an analyst's knowledge and belief about the data 
generation process of the problem at hand, can a specific causal effect be 
identified? As such, we emphasize that DAGs are great tools for a modeler to 
encode their assumptions about a problem, and not necessarily a guide on how 
to estimate a causal effect of interest. However, DAGs come with a set of 
testable implications, and incorporating them in any causal analysis adds 
robustness and defensibility to one's analysis. 




\blindtext[2]