\section{Overview of Causal Graphs}

Causal diagrams and causal graphical models have been introduced by Pearl 
(2000) as a powerful tool for causal inference, especially in observational 
studies. Perhaps one of the most important and useful features of causal 
graphs when dealing with causal inference problems is the clear illustration 
of the causal relationships between the variables. 

The goal of this section is not to provide a detailed overview of DAGs. Interested readers can refer to Pearl (2000) for a thorough introduction. 
Instead, we focus specifically on illustrating the power of DAGs to represent 
and encode complex  causal relationships between variables in an intuitive and 
clear manner. 

Consider the causal graph represented in Figure bla. Suppose we're interested 
in the effect of Z on Y. What the graph in figure bla implies is that Z is 
independent of Y(Z) given X, or in other words, the assignment mechanism is 
ignorable conditional on the values of X [TODO: insert equations for the 
ignorability assumption]. In such situations, it is sufficient to control for 
X to obtain an unbiased estimate of the causal effect of Z on Y. Now suppose 
we have a third variable, S, that affects both Z and Y (like X) but is 
unobserved (Figure bla). Now, the ignorability assumption of Z does not hold 
if we only condition on X, and we risk obtaining biased estimates of the 
causal effect of Z on Y if we fail to account for S. 


It is important to note that the graphical approach to causality focuses 
primarily on issues of identification of causal effects, that is, given a 
directed acyclic graph (DAG) that encodes an analyst's knowledge and belief 
about the data generation process of the problem at hand, can a specific 
causal effect be identified? As such, we emphasize that DAGs are great tools 
for a modeler to encode their assumptions about a problem, but not necessarily 
a guide on how to estimate a causal effects of interest. However, DAGs come 
with a set of testable implications, and incorporating them in any causal 
analysis adds robustness and defensibility to one's analysis. 

TODO: insert figures, add equations. 




\blindtext[2]