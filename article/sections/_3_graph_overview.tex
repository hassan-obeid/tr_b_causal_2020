\section{Overview of Causal Graphs}
\label{sec:graph-overview}

Causal diagrams and causal graphical models have been introduced by \citet{pearl_causality_2000} as a powerful tool for causal inference, especially in observational studies.
Perhaps one of the most important and useful features of causal
graphs when dealing with causal inference problems is the clear illustration
of the causal relationships between the variables.
While a formal introduction to the topic of directed acyclic graphs (DAGs) is beyond the scope of this chapter,
here we focus specifically on illustrating the power of DAGs to represent
and encode complex  causal relationships between variables in an intuitive and
clear manner.
Interested readers can refer to \citet{pearl_causality_2000} for a thorough introduction.

Consider the causal graph represented in Figure \ref{fig:simple-graph}.
Suppose we're interested in the effect of Z on Y.
What the graph in Figure \ref{fig:simple-graph} implies is that Z is
independent of Y(Z) given X, or in other words, the assignment mechanism is
ignorable conditional on the values of X.
In such situations, it is sufficient to control for
X to obtain an unbiased estimate of the causal effect of Z on Y.

\begin{figure}[h!]
   \centering
   \includegraphics[height=0.5\textwidth]{simple-graph.pdf}
   \caption{Simple DAG of the causal relationships between X, Y and Z}
   \label{fig:simple-graph}
\end{figure}

In comparison, note the set of structural equations needed to convey the same assumptions as Figure \ref{fig:simple-graph}.
\[Z = f_Z(X, \epsilon_Z)  \]

\[Y(z) = f_Y(X, z, \epsilon_Y)  \]

Now consider the case where there exists another latent confounding variable, U, which also affects both the treatment Z, as well as the outcome, Y.
Figure \ref{fig:simple-graph-confounded} illustrates this assumption in DAG form, and the equations below are the structural equation equivalent:

\[Z = f_Z(X, U, \epsilon_Z)  \]

\[Y(z) = f_Y(X, U, z, \epsilon_Y)  \]

\begin{figure}[h!]
   \centering
   \includegraphics[height=0.5\textwidth]{simple-graph-confounded.pdf}
   \caption{Simple DAG of the causal relationships between X, Y, Z and confounder U}
   \label{fig:simple-graph-confounded}
\end{figure}

It is clear from Figure \ref{fig:simple-graph-confounded} that conditioning on X alone does not block all the indirect paths between Z and Y, and thus ommitting it will yield biased results of the causal relationship between Z and Y.
The structural equations also show that the ignorability assumption
of Z does not hold if we only condition on X, and we risk obtaining biased estimates of the causal effect of Z on Y if we fail to
account for U.
Even in this very simple example with only three or four variables, one can see the advantage of using a causal graph to encode assumptions.
This advantage of DAGs becomes even more apparent in larger problems with many covariates available, and where the causal structure of the data is way more complicated, which are the types of problems typically faced by transportation demand modelers.

Another benefit of DAGs is that they come with a set of testable implications, and incorporating them in any causal
analysis adds robustness and defensibility to one's analysis.
We discuss those implications in section 5, and illustrate how to use those tests in one's analysis.

Lastly, it is important to note that the graphical approach to causality focuses
primarily on issues of identification of causal effects, that is, given a
directed acyclic graph (DAG) that encodes an analyst's knowledge and belief
about the data generation process of the problem at hand, can a specific
causal effect be identified?
As such, we emphasize that DAGs are great tools
for a modeler to encode their assumptions about a problem, but not necessarily
a guide on how to estimate a causal effects of interest.

\subsection{Prior Uses of Causal Graphs in Choice Modelling}
\label{sec:choice-graphs}

In our last subsection, we reviewed the basics of causal graphs: what are they and why are they useful?
We targeted that subsection at choice modellers who are unfamiliar with these tools.
However, readers should be aware of the history of causal graphs in choice modelling.
Indeed, choice modellers have used causal graphs (in limited fashion) for years.
Here are three examples to illustrate our point.
First, consider the case of Random Utility Maximization (RUM) models.

\begin{figure}
   \centering
   \includegraphics[width=0.25\textwidth]{rum-causal-graph}
   \caption{Archetypical RUM causal diagram}
   \label{fig:example-graph-rum}
\end{figure}

For decades, choice modellers have used stylized DAGs to depict RUM models.
In particular, these diagrams illustrate an assumed choice-making (i.e., causal) process.
As an example, see Figure \ref{fig:example-graph-rum}, reproduced from Figure 1 of \citet{ben_2002_integration}.
\citeauthor{ben_2002_integration} draw a two-part process.
First, they assume that explanatory variables cause unobserved utilities.
Then, they assume that the unobserved utilities cause the choice.

Note that although RUM diagrams adequately show the choice process, we still call them stylized.
Specifically, these diagrams lack detail about the relationships between explanatory variables.
When speaking collectively, we cannot tell if one explanatory variable causes another.
Unfortunately, as shown in Section \ref{sec:graph-importance}, such knowledge is crucial.
Without more detailed causal knowledge, our inferences may be inconsistent and arbitrarily bad.

\begin{figure}
   \centering
   \includegraphics[width=0.5\textwidth]{iclv-causal-graph}
   \caption{Archetypical ICLV causal diagram}
   \label{fig:example-graph-iclv}
\end{figure}

Another area of relation with prior choice modelling literature is in Integrated Choice and Latent Variable (ICLV) models.
These models make inference on one or more latent variables, alongside the typical parameters in one's model.
An example of such ICLV models is in Figure \ref{fig:example-graph-iclv}, based on Figure 5 of \citet{ben_2002_integration}.
As with the causal graphs of RUM models, ICLV causal graphs depict a particular set of structural modeling assumptions about the data generating process.
Here concern typically centers around unobserved mediators, i.e., variables caused by one's measured covariates that also influence one's outcome.
Omitting these variables leads to inconsistent estimates for one's remaining parameters, thus leading researchers to care deeply about ICLV models that can overcome such problems.

\begin{figure}
   \centering
   \includegraphics[width=0.5\textwidth]{sacsim-model-graph}
   \caption{Archetypical causal diagram for activity-based models}
   \label{fig:example-graph-abm}
\end{figure}

Finally, consider activity based travel demand models.
These models often come with a causal diagram that depicts the interrelations between the outcomes in the model (e.g., household location choice, destination choice, travel mode choice, departure time choice, route choice etc.)
For example, see Figure \ref{fig:example-graph-abm}, from \citet[Fig.1]{bradley_2010_sacsim}.
The purpose of such graphs is to explain the structure of the entire system of outcome models.
In particular, these graphs depict the researcher's assumed causal ordering of which outcomes temporally precede other outcomes.
Diagrams of activity-based models also show the modeller's assumptions about how considerations of future choices affect temporally-preceding decisions.

In RUM, ICLV, and activity-based model diagrams, the main distinction between the causal graphs of \citet{pearl_1995_causal} and the causal diagrams in econometrics are two two-fold.
First, causal graphs in RUM, ICLV, and activity-based models ignore relations between the explanatory variables.
Typically, choice modelling causal graphs depict all explanatory variables together as if they are in independently generated groups that then affect one's utilities and choices.
In the language used by a large group of causal inference scholars, econometric causal diagrams ignore the treatment assignment mechanism.

Secondly, causal graphs in choice modelling papers are purely didactic.
They convey how choice modellers perceive the world and the choice generation process.
However, they are seldom treated as a fully fledged model that has empirical implications that merit verification (e.g. conditional independence implications).
In this sense, choice modellers ignore the efforts from causal inference researchers in computer science.
There, researchers spend much effort testing their causal graphs to see if the data supports their graphs implications.

In conclusion, choice modellers have long made use of causal graphs in select contexts to convey causal assumptions about choice processes.
Thus far, however, we have underutilized these tools.
We seldom use causal graphs to encode assumptions about how our explanatory variables came to be in choice situations, and we do not routinely test causal graphs against empirical choice data.
These two issues represent opportunities for the field of choice modelling to gain from the insights and work of those who study causal inference.
The rest of the chapter will focus on how we can construct causal diagrams that pay attention to both explanatory and outcome variables, how we can test a given graph against one's data, and how one can deal with ``real-world'' graphs which frequently contain some sort of unobserved confounding.
