\section{Overview of Causal Graphs}

\subsection{Relations to Prior Literature in Choice Modelling}

With the introduction to causal graphical models given above, some readers may get the impression that we think the entire notion of a causal graph is foreign to traditional choice modelling.
We do not think this is true.
Here are two examples to illustrate our point.
First, consider the case of Random Utility Maximization (RUM) models.

For decades now, choice modellers have used stylized directed graphical models to denote or convey the meaning of their economic framework's notion of causality in the context of their research.
As an example, take the following figure from Walker and Ben-Akiva \cite authors.
In Figure \ref{figure_title} Walker and Ben-Akiva depict the assumption that covariates X cause random utilities U which collectively cause the choice via utility maximization.
The reason I call such diagrams stylized is because of their lack of detail of the relationships within the columns of the design matrix X.
When speaking of the covariates collectively, one is unable to distinguish whether one covariate causes another.
Such intra-covariate relationships are important, as demonstrated in Section \ref{Sec_label}.

Another area of relation with prior choice modelling literature is in Integrated Choice and Latent Variable (ICLV) models.
These models make inference on one or more latent variables, alongside the other parameters in one's model.
An example of such ICLV models is in Figure \ref{}, based on Figure __bla__ of Walker and Ben-Akiva \cite.
As with the causal diagrams for the RUM models, these causal diagrams are another set of graphical models that depict a certain set of structural modeling assumptions, a set of assumptions about the data generating process.
