\section{(Observable) Testing of Causal Graphs}
\label{sec:graph-testing}

\subsection{Description}
In the last section, we reviewed a process for creating an initial causal graph using expert opinion.
Critically, after drafting a causal graph, we should immediately test it against empirical data.
This is important because, if our graph captures inaccurate assumptions about the data generating process, then we have no reason to think that our conclusions from using the graph will be accurate.

To test our causal graphs against data, we will first test the implications of our graph that involve observable variables only.
We will defer the task of testing implications that involve unobserved / latent variables to the end  of this subsection.
For now, recall our discussion in Section \ref{sec:graph-overview} about the two basic implications of causal graphs: marginal independence and conditional independence.
In both cases, direct testing of marginal or conditional independence amongst nodes in the causal graph may be difficult.
Indeed, there are no direct tests of conditional independence that can detect all types of dependence, especially for continuous variables \citep{bergsma_2004_testing, shah_2020_hardness}.

As a result of this hardness, there are a myriad of research efforts aimed at testing conditional independence of two variables $X$ and $Y$, given a third variable $Z$ and any number of assumptions about the variables or the test statistic itself.
Some researchers proceed under the assumption that one has access to an approximation of the conditional distribution of $X \mid Z$ \citep{candes_2018_panning, berrett_2019_conditional}.
Other researchers designed conditional independence tests for general cases, assuming smoothness of the underlying data distributions and assuming accurate estimation of the distribution of the test statistic under the null hypothesis of conditional independence (e.g. \citet{zhang_2012_kernel, strobl_2019_approximate}).

Different from (but not excluding) these approaches, we will take an easier and less decisive route.
If a pair of variables have conditionally or marginally independent distributions, then their statistical moments will also be conditionally or marginally independent.
Instead of testing for marginal or conditional independence in distribution, we will perform a more tractable test for marginal or conditional independence in means.
If the variables in question are not conditionally or marginally independent in their means, then we know they are not independent in their distributions.
Conversely, even if a set of variables are marginally or conditionally independent in their means, this \textbf{does not} imply that the variables are independent in distribution.

This approach of indirectly assessing distributional independence by testing mean independence is not new.
The following papers have all proposed and implemented such an idea: \citet{burkart_2017_predictive, chalupka_2018_fast, inacio_2019_conditional}.
For conditional independences, the crux of the approach is to predict $Y$ based on $X$ and $Z$, then compare against a prediction of $Y$ based on a resampled value of $X$ and the original $Z$.
If $Y$ is mean-independent of $X$ given $Z$, i.e. $E \left[ Y \mid X, Z \right] = E\left[ Y \mid Z \right]$, then the predictive power of a model with resampled $X$ should resemble the predictive power of a model with the original $X$.
After all, in both cases, the conditional expectation of $Y$ is independent of one's $X$ values (real or resampled).
When assessing marginal independencies, one removes $Z$ from the models for the expectation of $Y$ and proceeds as described.

Note that as with the case of testing distributional independence, testing mean independence still requires researchers to make choices.
We have to select models for $E \left[ Y \mid X, Z \right]$ and $E\left[ Y \mid Z \right]$, respectively, for testing conditional and marginal mean-independence.
We also have to choose the performance statistic (e.g. $R^2$, log-likelihood, etc.) to compare these models.
Lastly, we also have to select a resampling method.
In particular, how (if at all) will our resampling strategy account for the possible dependence between $X$ and $Z$?

% State the choices made in this work.
% Provide some discussion of the alternatives and the effects of different choices.
For our demonstration, we made the following choices.
First, we used linear regressions to model $E \left[Y \mid X, Z \right]$ and $E \left[ Y \mid Z \right]$.
Second, we chose to use $R^2$ as our test statistic for judging the predictive performance.
Third, we have chosen to resample the $X$ vectors without replacement, keeping the length of the resampled vector equal to the length of the original vector.
In other words, we permute $X$.
And finally, we have chosen to visualize these tests by plotting a vertical line to display our observed test-statistic and by plotting the distribution of our test statistics based on the permutations of $Y$.

Our rationale for these choices are as follows.
In our dataset, most of our explanatory variables were continuous (at least in theory).
Accordingly, $R^2$ seemed a sensible performance metric for a model of the conditional expectation of a continuous random variable.

In contrast to our choice of performance metric, we chose our conditional expectation models and resampling methods based on empirical testing.
In particular, we created simulations to assess our mean-independence testing procedure.
We assessed the performance of our mean-independence testing procedures using simulations where $Y \leftarrow Z \rightarrow X$ and $X$ either did or did not cause $Y$.

Of particular importance were our simulations under the null hypothesis where $X$ was conditionally independent of $Y$.
Our initial simulations used random forests as our conditional expectation models and permutations as resampling methods.
Random forests are a non-parametric method that would allow us to have less fear of model misspecification, and permutations are easy to implement.
However, under the null hypothesis, we discovered that the tests based on the random forest models did not result in p-values that were uniformly distributed.
Given that we planned to use these procedures in a manner akin to hypothesis-testing, we hoped that our test statistics would be U-statistics.
When we switched from the combination of random forests and permutations to linear regressions and permutations, our p-values were indeed empirically, uniformly distributed.
Moreover, we still retained high power.

We do not claim that these choices for assessing mean independence will always be appropriate.
Indeed, one should assess one's tests on simulated data that resembles one's real data.
For our illustrative purposes though, the combination of linear regressions, permutations, and $R^2$ resulted in adequate tests of marginal and conditional mean-independence.

% Describe latent variable testing
Now that we have described testing with observable variables, we can more easily describe conditional independence tests that involve unobserved (i.e., latent) variables.
Indeed, when dealing with observational data, we will frequently find ourselves not having observed all variables that are of interest.
Nevertheless, we still wish to test whether our data contradicts our graph.
One way to directly extend our conditional independence testing to account for latent variables is to adopt a missing data perspective and impute the latent variables from a prior distribution.
In particular, we can generalize our previous tests as follows.

First, we can consider that instead of performing one test with a set of observed $X$, observed $Y$ and observed $Z$, we instead perform tests of observed $X$, observed $Y$, and imputed $Z$.
This recasts the randomness underlying the null-distribution in our original test statistic of $R^2$ as a function of our permutation of $Y$ and our imputation of $Z$.
Here, we impute $Z$ by sampling from the prior or posterior distribution of $Z$, depending on whether we're testing independencies before or after performing inference on our model's parameters.
Moreover, we'll now compute the p-value of this test statistic by marginalizing over the permutations and imputations.
Specifically, our p-value will be $\mathcal{E}_{\textrm{samples, permutations}} \left[ 1_{R^2 \left( X, Y, Z_{\textrm{sampled}} \right) <  R^2 \left( X_{\textrm{sampled}}, Y_{\textrm{sampled, permuted}}, Z_{\textrm{sampled}} \right)} \right]$.
For reference, this is the same as the p-value for test statistics (or discrepancies) defined in \citet[Eq. 7]{gelman_1996_posterior}.

Lastly, because our test statistic is itself a random variable that depends on the imputed values of $Z$, we must change our visualization method.
Instead of plotting a single line versus a distribution, we now plot two distributions against one another.
We first plot the distribution of ``observed'' test statistics that is computed using the observed $X$, observed $Y$ and imputed $Z$ values.
Then, we plot the distribution of ``sampled'' test statistics using prior samples of $\left( X, Y, Z \right)$ as a reference.
For an example of such a plot, see Figure \ref{} (discussed in the following subsection).
Note, we have one value per imputed vector $Z$ in both the observed and sampled distributions.
However, for the distribution of ``sampled'' test statistics, we marginalize over permutations of $Y$ since this distribution represents the null hypothesis of conditional or marginal independence.

Of course, since there is generally ``no free lunch,'' this approach comes with drawbacks.
One major concern is that such tests are very sensitive to assumptions about the joint prior distribution, $P_{\textrm{prior}} \left( X, Y, Z \right)$
In our experience using simulation, if the observed data $\left( X, Y \right)$ is unlikely under the joint prior distribution $P_{\textrm{prior}} \left( X, Y, Z \right)$ then the conditional independence test is likely to fail.
As always, the failing test indicates that the observed data is unlike the data simulated and used to construct the reference distribution.
However, in this case, the dissimilarity may not relate to the conditional independence properties but may instead be an artifact of the sampled $\left( X, Y \right)$ being generally dissimilar to the observed $\left( X, Y \right)$.
This highlights the need for extensive prior predictive checking of one's assumed joint prior, $P_{\textrm{prior}} \left( X, Y, Z \right)$, \textit{before} using conditional independence tests on causal graphs with latent variables.

\subsection{Demonstration}
% Illustrative causal graph
To demonstrate the testing procedures described above, we used the causal graph in Figure \ref{fig:graph-for-testing}.
This causal graph shows a set of hypothesized causal relationships between variables thought to contribute to the utility of the drive-alone travel alternative in our dataset.
As drawn, this graph encodes multiple marginal and conditional independence assumptions.
Tools such as Daggity \citep{textor_2016_robust} can be used to automatically infer all independencies based on one's graph.
For didactic purposes, however, we focused our attention on two particular independence assumptions.

First, we tested the assumption of marginal independence between the number of licensed drivers and the number of automobiles in a household.
A-priori, we may expect this independence to be have low-probability since the number of automobiles may generally be positively related to the number of licensed drivers in a household.
Secondly, we tested the assumption of that travel cost was independent of travel time, conditional on travel distance.
Unlike the first independence that we focus on, this conditional independence assumption is a-priori more credible.
Despite their differing levels of a-priori credibility, we will see how we can test both using the procedures described in the previous subsection.

% Marginal testing results
% \begin{figure}
% \centering
% \includegraphics[width=0.5\textwidth]{exa}
% \caption{Example decision tree for bicycle consideration}
% \label{fig:marginal-independence-test}
% \end{figure}

In particular, Figure \ref{fig:marginal-independence-test} shows the results of using permutation, linear regression, and $R^2$ to test the hypothesis of marginal independence between the number of automobiles and the number of licensed drivers in a household.
The empirical p-value of 0 confirms that the observed data is unlikely given the null-hypothesis of marginal, mean-independence.
More specifically, when regressing the number of licensed drivers in a household on the number of cars in that household, one achieves an $R^2$ near 0.4.
In contrast, when permuting the number of cars in the household and re-estimating the regression, the distribution of p-values concentrates around 0.
This plot visualizes the fact that---through the lens of our chosen test statistic ($R^2$), linear regression model, and permutation-based resampling strategy---data generated under an assumption of marginal mean-independence does not ``look like'' the observed data.
Accordingly, we should consider the weaker assumption of marginal dependence, where we can make data-generating assumptions that offer greater realism and concordance with our observations.

% Conditional testing results
% \begin{figure}
% \centering
% \includegraphics[width=0.5\textwidth]{exa}
% \caption{Example decision tree for bicycle consideration}
% \label{fig:conditional-independence-test}
% \end{figure}
In Figure \ref{fig:conditional-independence-test}, we have an analogous visualization of a conditional independence test.
Here, we test the hypothesis that travel time is mean-independent of travel cost, conditional on travel distance.
To execute this test, we model the conditional expectation of travel time as a linear function of travel cost and travel distance.
As with the marginal independence test results, the $R^2$ of the model using the observed values of travel cost are greater than the model's $R^2$ using any simulated datasets.
Again, this means that the $R^2$ using observed data is unlikely given our method of sampling from the null distribution of $R^2$ given conditional, mean-independence of travel time and travel cost.

As before, we should respond to this result by considering conditional dependence between our variables of interest.
In particular, we should investigate the idea that travel time is associated with travel cost, conditional on travel distance.
Why might this be the case?
Does travel time cause travel cost when driving alone?
Does travel cost cause travel time while driving alone?
Does some other set of variables (potentially unmeasured) cause both travel time and travel cost?

Thinking through these questions, we can immediately think of latent variables that cause both travel time, travel cost, and the choice.
For example, consider whether one drives alone over the San Francisco Bay Bridge.
If one crosses the bridge, then traffic delays will likely increase one's travel time.
Moreover, if one crosses the bridge, then one's travel cost is higher due to tolls that one must pay.
And finally, if one takes a toll lane to get across the bridge faster, then one also pays a higher price.
Overall, we can think of intuitive explanations for why there might be a dependence between travel time and travel cost, even after conditioning on travel distance.
These explanations suggest particular relationships to analyze and particular variables, such as bay bridge crossings, to include in one's mode choice (i.e. outcome) model.

\subsection{Additional techniques}

The methods presented in this section have focused on testing independence assertions using one's dataset.
While this approach may be the most accessible strategy for testing one's causal graph, there are other applicable techniques.
For instance, \citet{pitchforth_2013_proposed} put forth a detailed checklist of qualitative questions for one's causal graph.
Answering these questions should increase the trustworthiness of one's graph.
Alternatively, there are other quantitative tests of one's causal graph that were not explored in this section.

For instance, a causal graph makes assumptions about the number of independent variables in one's dataset: i.e., the nodes in one's graph without parent-nodes.
One can test this assumption by estimating the ``intrinsic dimension'' of one's data, and testing whether the intrinsic dimension equals the number of independent variables implied by one's graph.
For more information on estimating the intrinsic dimension of a dataset, see \citet{camastra_2016_intrinsic, song_2019_identification}.
See \citet{chenwei_2019_likelihood} for an extension of this idea for the situation where one suspects or cannot rule out unobserved confounding.

% Test vanishing tetrads / t-separation
Another empirical implication of one's causal graph is the existence of so-called ``vanishing tetrads'' \citep{spearman_1904_general}.
This term signifies that the difference between the product of two particular pairs of covariances must be zero.
As stated, this implication of one's causal graph is hard to intuitively understand.
However, one can graphically determine the existence of vanishing tetrads and determine which variables are part of these tetrads.
Once one has determined the tetrads that should vanish according to one's graph, one can estimate the corresponding covariances and test to see if their difference of products is indeed unlikely to be zero.
Such a test is yet another way to empirically determine whether one's graph is incompatible with one's dataset.
For the original theorems proving that tetrads can be graphically identified and characterized, see \citet{shafer_1996_vanishing} and references therein.
For a more detailed and intuitive explanation of the graphical criterion for vanishing tetrads, see \citet{thoemmes_2018_local}.

% Test functional inequalities / constraints, i.e. entropy constraints / inflation technique
Finally, we note that there are still a whole host of other techniques for testing one's causal graphs.
Many of these remaining techniques are useful when one's causal graph contains unobserved (i.e., latent) variables.
On one hand, we can use ``triad constraint'' tests that test independence between ``pseudo-residual'' values and one's explanatory variables \citep{cai_2019_triad}.
Results from these tests are particularly useful for judging assumptions about how unobserved variables in our graph relate to each other and to our observed variables.

Relatedly, one can make use of constraints on entropies of our observed variables instead of independencies.
The basic idea is that differing assumptions about the structure of the unobserved variables in our graph imply differing amounts of entropy in the variables that we do observe, so we should test for these differences in entropy.
For more information and examples, see the literature about:
\begin{itemize}
  \item inequality constraints, e.g. \citet{tian_2002_testable, kang_2006_inequality, ver_2011_sequence}
  \item information inequalities, e.g. \citet{chaves_2014_inferring}
  \item entropic inequalities, e.g. \citet{chaves_2014_causal}
  \item the inflation technique, e.g. \citet{wolfe_2019_inflation, navascues_2020_inflation}
\end{itemize}
