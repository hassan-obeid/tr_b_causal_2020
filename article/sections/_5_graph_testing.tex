\section{Testing of Causal Graphs}

In the last section, we reviewed a process for creating an initial causal graph using expert opinion.
Critically, after drafting one's causal graph, we should immediately test the graph against empirical data.
If our graph captures inaccurate assumptions about the data generating process, then we have no reason to think that our conclusions from using the graph will be accurate.

To test our causal graphs against data, we will test the implications of our graph.
Specifically, we noted in Section \ref{sec:graph-overview} that causal graphs express two basic implications: marginal independence and conditional independence.
In both cases, direct testing of marginal or conditional independence amongst nodes in the causal graph may be difficult.
Indeed, there are no direct tests of conditional independence that can detect all types of dependence \citep{shah_2020_hardness}.

Instead, we will take an easier but less decisive route.
If a pair of variables have conditionally or marginally independent distributions, then their statistical moments will also be independent.
So instead of testing for independence in distribution, we will perform a more tractable test for independence in means.
If the variables in question are not conditionally or marginally independent in their means, then we know they are not independent in their distributions.
Conversely, even if a set of variables appear to marginally or conditionally independent in their means, this \textbf{does not} imply that the variables are independent in distribution.
