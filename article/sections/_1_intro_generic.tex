\section{Introduction}
\label{sec:intro}

The following chapter is concerned with the existing disconnect between the 
fields of travel demand modeling and causal inference. 
More specifically, the chapter is motivated by the current lack of use of methods and findings from 
the causal inference literature in travel demand modeling. 

More often than not, the development of behavioral models in transportation, 
specifically transportation demand models, is driven by the need to evaluate 
the impact of external interventions, in the form of alternative policies, on a 
certain outcome of interest. 
These questions that transportation demand models 
are built to answer are causal by definition: we are interested in how the 
system reacts to \textit{external} interventions. 
Yet, when those models are 
developed, there is very little consideration given to causality, and when 
causal concepts are accounted for, the process is done implicitly without a 
formal framework. 

While the field of transportation demand modeling could benefit greatly from 
incorporating causal inference techniques, there are barriers that have made 
this integration slower than what one would hope. 
These barriers stem from the difference between the types of problems transportation demand modelers deal with and those that are typically studied in the causal inference literature. 
Perhaps the main fundemental difference is that demand modelers 
are typically trying to forecast the impacts of policies that haven't been 
implemented or seen before, which requires additional work and a change to the
typical causal modeling workflow in order to translate a given policy 
(treatment) into a set of characteristics and variables that exist in the
data and system at hand (please refer to \citet{brathwaite_2018_causal} for a 
more thorough discussion of those barriers). 
While those barriers make the problem of demand modelers harder, there is still a lot to gain from 
incorporating causal inference techniques where appropriate, and to contribute
in turn where the literature lacks.  

The relevance of this topic now is motivated by the significant boost in the causal inference literature recently, both in the potential outcomes and the causal graphical modeling frameworks.
The goal of this chapter is to formalize a workflow for approaching 
transportation demand modeling problems from a causal perspective. 
We will draw heavily on the use of directed acyclic graphs (DAGs) formalized by \citet{pearl_causality_2000} as a means of representing the modeler's knowledge and assumptions about a given problem. 
The chapter will provide an overview of DAGs, the 
testable implications that come with one's causal representation, the main 
tests that one could do to falsify or justify a given causal graph, and then how 
to use a causal graph to estimate the causal relationships of interest. 
We will demonstrate the use of this framework through simulations, where we 
clearly show the benefits and implications of this approach as opposed to 
traditional approaches. 
The last part of this chapter deals with the more complicated issue of latent 
confounding, where one variable confounds two or more variables in the causal 
graph. This type of confounding creates variations in the outcome variable that are not caused by 
the confounded variables but are correlated with it, which biases the estimated 
causal effects of those variables if nothing is done to account for the 
confounding. 
We focus on a recent technique to deal with latent confounding suggested by \citet{wang_2019_blessings} to address unobserved confounding when collecting additional data is not feasible. 
We describe this problem in more details in section 7.
