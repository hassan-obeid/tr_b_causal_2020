\section{Introduction}
The following chapter is concerned with the existing disconnect between the fields of travel demand modeling and causal inference. More specifically, the chapter is motivated by the current lack of use of methods and findings from the causal inference literature in travel demand modeling. The field of causal inference has had a significant boost in interest recently with a new direction of research based on the potential outcome framework and causal graphical models as put forward by Pearl et al. Specifically we can leverage the deconfounder work by Blei et al. to address unobsrved confounding in tranportation applications.

More often than not, the development of behavioral models in transportation, specifically transportation demand models, is driven by the need to evaluate the impact external interventions, in the form of alternative policies, on a certain outcome of interest. These questions that transportation demand models are built to answer are causal by definition: we are interested in how the system reacts to \textit{external} interventions. Yet, when those models are developed, there is very little consideration given to causality, and when causal concepts are accounted for, the process is done implicitly without a formal framework. 



\blindtext[2]

This is an example citation \cite{brathwaite_2018_causal}
