\section{Introduction}
\label{sec:intro}

In transportation, we often build models for policy evaluation.
Specifically, we often use behavioral models to evaluate the impact of external interventions on travel outcomes.
These evaluations are explicitly causal: we are interested in how the
system reacts to \textit{interventions}.
Yet, when these models are
developed, causality is often not considered. Furthermore, when
causal concepts are accounted for, the process is done implicitly without a
formal framework.

This chapter aims to fill this gap.
Specifically, we focus on addressing the existing disconnect between the fields of travel demand modelling and causal inference.
The chapter is motivated by the current lack of use of methods and findings from
the causal inference literature in travel demand modelling and choice modelling more broadly.

While the field of transportation demand modelling could benefit greatly from
incorporating causal inference techniques, there are barriers that have slowed this integration.
These barriers stem from the difference between the types of problems transportation demand modellers deal with and those that are typically studied in the causal inference literature.
Perhaps the fundamental difference is that demand modellers are typically trying to forecast the impacts of policies that haven't been implemented or seen before.
Forecasting the effects of unseen interventions requires additional work and a change to the typical causal modelling workflow.
In particular, we must translate a given policy (treatment) into a set of characteristics and variables that exist in the data and system at hand.
(For a more thorough discussion of this and other barriers, please refer to \citet{brathwaite_2018_causal}.)

While these barriers complicate the efforts of demand modellers, there is still a lot to gain from
incorporating causal inference techniques where appropriate and from contributing
to the causal inference literature where it's lacking.

This topic is even more relevant now because of the significant and recent boost in the causal inference literature, both in the potential outcomes and the causal graphical modelling frameworks.
Leveraging these latest advances, the goal of this chapter is to formalize a workflow for approaching
transportation demand modelling problems from a causal perspective.
We will draw heavily on the use of directed acyclic graphs (DAGs) formalized by \citet{pearl_causality_2000} as a means of representing the modeller's knowledge and assumptions about a given problem.
The chapter will provide an overview of DAGs, the
testable implications that come with one's causal representation, the main
tests that one could do to falsify or justify a given causal graph, and then how
to use a causal graph to estimate the causal relationships of interest.
We will demonstrate the use of this framework through simulations, where we
clearly show the benefits and implications of this approach as opposed to
traditional approaches.

The last part of this chapter deals with the more complicated issue of latent
confounding, where an unobserved variable confounds two or more variables in the causal graph.
Confounding creates variations in the outcome variable that correlate with but are not caused by the treatment variables.
These spurious correlations bias the estimated causal effects if nothing is done to account for them.
In Section \ref{sec:latent-confounding}, we focus on a recent technique by \citet{wang_2019_blessings} for addressing unobserved confounding when collecting additional data is not feasible.
Finally, in Section \ref{sec:discussion}, we briefly discuss the many subsequent and important steps that are necessary for making and benefiting from one's causal inferences.
