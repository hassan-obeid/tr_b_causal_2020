\section{Latent Confounding}

\subsection{Latent Variable Models}

Latent variables have been used time and again in different applications of statistical 
and econometric models in different fields, ranging from epidemiology to transportation.
A latent variable is defined as a covariate not directly observed about subjects in the sample.
Examples of these variables include measurements about happiness, lifestyle, economic expectation, 
morale, among others in many different fields.
Latent variable models are based on the assumption that observed variables are independent 
conditional on latent variables.
Examples of these models include Integrated Choice and Latent Variable Models, Generalized Linear Latent and Mixed Models, Factor Analysis Models, and Latent Class Models.
Latent variable models usually rely on collecting indicator/proxy data to infer the levels of latent variable for individuals in a certain sample.
The collection of this data is not always easy (due to potential sensitivity of subjects towards some attitudinal questions)
or even feasible. 
This makes it difficult for researchers to collect necessary information that allows
for controlling for unobserved factors affecting one's likelihood of choosing an available option in their choice set.

TODO: Mechanics of how latent variable models control for unobserved confounding.
