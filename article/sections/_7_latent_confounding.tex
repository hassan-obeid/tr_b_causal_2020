\section{Latent Confounding}

This section focuses on the more complicated and realistic case that is 
typically faced by demand modelers, where we have latent confounders that 
affect the treatment assignment of our causal variables of interest. We first
go over a few examples of confounding in transportation analysese, explain 
the challenges that come with such cases, and present a few approaches to 
dealing with this issue, specifically the recent de-confounder technique of 
Wang et al. (2019). Our application will show how directed acyclic graphs can 
help increase transparency about one's reasoning regarding the number of 
confounders, assumptions for which variables are confounded, and the models 
needed to estimate the causal effects. In addition to the application above, 
we use simplified simulation scenarios to further investigate the usefulness 
and pitfalls/sensitivity of this approach for generating accurate model 
estimates. 


\subsection{Examples of confounding}


Confounding occurs when a certain (confounding) variable induces variations in
both the outcome, as well as the treatment (policy) variable of interest, 
creating correlation between the treatment and the outcome that is not caused
by the treatment variable. When the confounding variable (or variables) is 
observed, we can control for the confounding effect, and there exists many 
methods in the literature for how to do that, including post-stratification,
multiple regression, propensity score methods, etc. It is when the 
confounding variable is unobserved that the problem becomes significantly 
more challenging. For an illustration, imagine we're interested [TODO: 
insert relevant transportation example with confounding]. Without 
accounting for the latent confounder, we risk getting very biased estimates 
of the effects of our treatment variable of interest, created by the 
induced variations by the confounder in both the treatment and the outcome 
variable. 



The problem of latent confounding and ommitted variable bias is widely 
acknowledged in the transportation literature, and demand modelers can draw on 
a list of method to account for confouding in some specific circumstances. 
Integrated choice and latent variable (ICLV) models are a way to account for 
the effect of unobserved attitudinal variables which may affect the selection 
of some individuals into some treatment level, as well as an outcome of 
interest. For example, a person's unobserved beliefs and attitudes towards 
being environmentelly friendly may both affect whether she chooses to bike to 
work, as well as whether she lives close to a bike infrastructure, which may 
bias our observational study if we're interested in the effect adding a 
bikelane has on the mode share of bicycles. Those methods, however, rely on 
collecting attitudinal indicators typically obtained by conducting (usually) 
expensive surveys with a sample of the people. This may not be an option in 
many cases [TODO: list some examples where it's challenging to conduct surveys 
with people to collect indicator questions]. 


\subsection{The deconfounder algorithm}



One recent method that has been proposed to deal with the problem of latent 
confounding is the deconfounder algorithm by Wang et al. (2019). The method 
attempts to control for the confouding variable by estimating a "subsitute 
confounder", a set of variables that once controlled for, renders all 
variation in the treatment variables of interest exogenous. The process of 
applying the method is actually quite straightforward and simple. It proceeds 
as follows:


	- First, estimate the substitute confounder using any good latent variable 
	model the modeler chooses. The authors suggest estimating a factor model 
	with k factors on the set of covariates the modeler is interested in. 

	- Second, check the factor model's accuracy using posterior predictive 
	checks (add details). 

	- Once a sufficiently accurate latent variable model is recovered, use it 
	to estimate an expected value of the latent variable for each observation, 
	and control for this value in the outcome model, alongside the treatment 
	variables and other covariates of interest. 



One main assumption of the deconfounder algorithm is that the data at hand 
should have multi-cause confounders, thus the title of the paper, "The 
blessings of multiple causes". In other words, this method works when the 
unobserved confounders affect multiple of the observed causes (or treatment 
variables) of interest, alongside the outcome. This assumption is weaker than 
the one required for ignorability to hold, which requires the absence of both 
single cause and multi-cause confounders for accurate causal inferences. 


[TODO: Add more mathematical formulations and formulas, and formulate the 
model rigorously]


\subsection{Case study: Simulation}


The purpose of this section is to investigate the effectiveness of the 
deconfounder algorithm (Wang et. al, 2019) in adjusting for unobserved 
confounding. We use a simulated mode choice data where travel distance 
linearly confounds both travel time and travel cost. We then mask the travel 
distance data and treat it as an unobserved variable. 


We estimate three models:


	- Model 1: A multinomial logit with the correct original specification, 
	EXCEPT we ommit the travel distance variable in the specification without 
	trying to adjust for it.

	- Model 2: We use the deconfounder algorithm to try to recover the 
	confounder (travel distance). In this method, we use all the variables in 
	each mode's utility to recover that mode's confounder.

	- Model 3: We use the deconfounder algorithm to try to recover the 
	confounder (travel distance), but this time, we only use travel time and 
	cost in the factor model, instead of all the variables in the utility 
	specification of each mode. This is different than what is suggested in 
	Wang et al., where they use all the observed covariates to recover the 
	deconfounder. 


We compare the estimates of the coefficients on travel time and cost to the 
true estimates used in the simulation. The main findings of this exercise are 
the following:


Using the true variables believed to be confounded (i.e. method 3 where only 
travel time and cost are used to recover the confounder) leads to a better 
recovery of the true confounder. This suggests that it may be better to run 
the deconfounder algorithm based on a hypothesized causal graph, rather than 
just running it on all the observed covariates.

Similar to what we found in the investigating_decounfounder notebook, the 
effectiveness of the deconfounder algorithm is very sensitive to small 
deviations in the recovered confounder. Although method 3 returns a relatively 
good fit of the true confounder, the adjusted coefficients on travel time and 
cost do not exhibit any reduction in the bias resulting from ommitting the 
true confounder, and the coefficients on the recovered confounder are highly 
insignificant. This raises questions about the usefulness of the deconfounder 
algorithm in practice.





Things to include:


- Simpson's paradox

- Introduction of deconfounder algorithm

- Challenges of deconfounder. 

- Simulation illustrations



\blindtext[2]

This is another example citation \cite{wang_2019_blessings}
