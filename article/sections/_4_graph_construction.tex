\section{Causal Graph Construction}

Construction of an initial causal graph typically proceeds as follows.
At a high level, we
\begin{enumerate}
   \item adopt a population perspective,
   \item brainstorm all variables that we think affect the system that generates our observations,
   \item remove any variables that would cause bias in our causal inferences,
   \item connect all variables in our graph according to our a-priori beliefs about causal relations amongst these variables,
   \item consider how the graph structure may differ between individuals and subgroups within the population.
\end{enumerate}
The following paragraphs will describe these steps in detail.

To begin, we adopt the position of a researcher concerned about population level relationships.
This means we will think through what is a likely generative model for all individuals.
Later on, we will devote time to thinking about how subgroups and individual heterogeneity may affect our causal graphs.

Now, we add our first variables to our graph, the outcome variable(s) of interest.
Then, we note all the variables that we believes to influence the variable(s) of interest.
We refer to these variables to as our initial explanatory variables.

Next, we iterate through the initial explanatory variables.
For each current explanatory variable in the iteration, we think of variables that may modify the effect of the current explanatory variable on the outcome(s) of interest.
We refer to these variables as effect modifiers.
Note that some effect modifiers may be a part of our initial explanatory variables.
For any effect modifiers that one thinks of, outside of the initial explanatory variables, add them to our causal graph.

After adding explanatory and effect modifying variables to the graph, we turn our attention to mediating variables.
Again, we iterate through each explanatory variable.
On each iteration, we brainstorm all the variables through which our current explanatory variable influences the outcome.
For instance, consider how the presence of a bike lane influences bicycle mode choice.
We hypothesize that an individual's subjective perception of safety is be the primary (or sole) variable through which bicycle lane presence influences mode choice.
Accordingly, we add subjective perception of safety to our causal graph for travel mode choice.

Coming to our second to last category of variables, we think of confounding variables.
The process is similar to how we generated effect modifying variables.
We iterate through each of the explanatory, mediating, and effect modifying variables, thinking specifically of any variables that both cause the current variable in the iteration and cause the outcome variable(s).
We call these variables, which cause our outcome and current variables in the iteration, confounding variables.
We will add each of them to our causal graph.

For the last set of variables, we should explicitly consider the role of time, even in research that may be cross-sectional due to the data that is available to us.
In reality, how do we think our system evolves over time?
If we consider multiple observations of a given decision maker, how do variables observed of that decision maker at time $t$ partially cause future variables important to the context or outcomes for that decision maker at time $t+1$?
How do the actions of a decision maker $i$ at time $t$ partially cause the future context or outcomes of a decision maker $j$?
We should add explicit nodes to our graph, subscripted or denoted by time, to show the cross-time causal relationships in our system.

At this point, we have added all the outcome, explanatory, effect modifying, mediating, confounding, and time-indexed variables that we can think of to our causal graph.
They are all disconnected nodes, singletons, in the graph.
We now focus on pruning nodes from this graph, before drawing our final hypothesized connections.
In particular, we focus on pruning ``post-outcome'' variables that are not part of the causal graph for future time periods or other observations.
The reason for this is that conditioning on such post-outcome variables biases one's causal effect estimates.

To remove the problematic variables, we iterate through each of the non-outcome variables in our graph, and we assess whether the variable is actually a result of the outcome (perhaps in combination with other variables in our graph).
These post-outcome variables temporally follow the outcome variable(s) but do not cause variables in the causal graph for other observations.
We remove all such post-outcome variables from our graph.

Now is a good time to step back and consider what other researchers have thought.
Specifically, we should conduct a literature review to see how other researchers have conceptualized the problem that we are working on.
Have they included variables that we have not?
Were those variables related our outcomes of interest?
If so, should we add these variables to our causal graph, and how?
Do the included variables of other researchers point suggest the existence of confounders in their work that we should include in our graph?
Have other researchers ascribed differing roles to our graph's current variables than we have?
For example, have other researchers judged a variable to be confounder, when we solely thought of the variable as an affect modifier?
Critically examine the evidence for these alternative decisions to see if we should also reconsider how we're judging our variables.

Finally, we need to connect the variables in our graph.
\begin{enumerate}
   \item Draw direct arrows from our explanatory variables, confounders, and effect modifiers to the outcomes.
   \item Draw arrows from the explanatory variables to the mediators, and then draw arrows from the mediators to the outcomes.
   \item Draw arrows from the confounders to the explanatory variables and mediators that they may cause.
   \item Draw arrows from the variables in time $t$ to the variables that they cause in time $t+1$.
\end{enumerate}
After drawing in all arrows, we should now have a fully connected causal graph.
Take a moment to look over the graph to ensure there are no remaining singletons and that we have not drawn any spurious connections.
Also, take a moment to celebrate.
Drawing one's first causal graph is hard work!

After celebrating, take a moment to pursue the following graph editing exercises.
First, think about how the graph might differ across sub-populations.
What sub-populations exist in your population of interest?
Are there any causal relationships that should not exist for a given sub-population?
For instance, are the outcomes in some sub-populations independent of a given explanatory variable?
Can you think of any inverted causal relationships in this sub-population?
(I.e., for a given sub-population $B \rightarrow A$ instead of $A \rightarrow B$?)
Consider adding these sub-population indices to one's initial causal graph, or if this is not clear enough, draw amended causal graphs for each sub-population.
Now, one can actually relax.
This concludes the ``purely mental'' drafting of one's causal graph.
In the next section, we'll look at testing this graph against data, and making any edits deemed empirically necessary.
