\section{Perils of disregard}
Consider the following scenarios: 1) Based on recent input from the public, the Department of Transportation (DOT) of a certain city is considering implementing a new parking policy that discourages parking in the Central Business District and therefore would encourage the use of more active transportation modes such as walking or biking.
2) A certain DOT is considering implementing a Streetcar system to incentivize more transit oriented development in a certain region. 

When thinking about these two "policy" proposals (funding the Streetcar system in the second case), it could that the DOT (or any other agency) in question ran analyses (either based on meticulous modeling or anectodal evidence) that allowed them to deduce that implementing such policies would directly result in their desired output or achieve their desired goal.
This inherently assumes the causal relationship between the policy and the final output (implementing the new parking policy will increase the share of active transportation modes in CBDs and funding the streetcar system will generate more transit oriented developments).
Moreover, these analyses - and conclusions from such analyses - of the impact of the proposed policies are also based on a certain belief of how the the world or the system works. 
However, in many instances (insert references to policies etc..) such belief of structure of the system is not made clear; we only hear about the policy and its most downstream impact. 
Such depiction of proposed policies maintains an obscure representation (at least to the knowledge of the average person) of the system and the interactions between its intermediate elements.

Directed Acyclic Graphs (DAGs) otherwise known as causal graphs allow one to clearly represent one's assumptions about the problem at hand.
DAGs have been used in fields ranging from epidemiology, scheduling, and network structures and have shown to be extremely useful. 

DAGs could show to be very useful in addressing policy questions (in our case transportation policy questions).
Brathwaite & Walker [insert citation] have illustrated an example of how DAGs can be used to answer such questions.
Brathwaite & Walker haven't showed an empirical application of how using DAGs can result in different model estimates.

We approach this empirical exercise by highlighting a simplified problem (although maybe not realistic).

\blindtext[2]